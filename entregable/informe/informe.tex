% ******************************************************** %
%              TEMPLATE DE INFORME ORGA2 v0.1              %
% ******************************************************** %
% ******************************************************** %
%                                                          %
% ALGUNOS PAQUETES REQUERIDOS (EN UBUNTU):                 %
% ========================================
%                                                          %
% texlive-latex-base                                       %
% texlive-latex-recommended                                %
% texlive-fonts-recommended                                %
% texlive-latex-extra?                                     %
% texlive-lang-spanish (en ubuntu 13.10)                   %
% ******************************************************** %


\documentclass[a4paper]{article}
\usepackage[spanish]{babel}
\usepackage[utf8]{inputenc}
\usepackage{charter}   % tipografia
\usepackage{graphicx}
%\usepackage{makeidx}
\usepackage{paralist} %itemize inline

%\usepackage{float}
%\usepackage{amsmath, amsthm, amssymb}
%\usepackage{amsfonts}
%\usepackage{sectsty}
%\usepackage{charter}
%\usepackage{wrapfig}
\usepackage{listings}
\lstset{language=C}
\usepackage{caption}
\usepackage{color}

% \setcounter{secnumdepth}{2}
\usepackage{underscore}
\usepackage{caratula}
\usepackage{url}
\usepackage[document]{ragged2e}
\usepackage[export]{adjustbox}
\usepackage{subcaption}
\usepackage{floatrow}


% ********************************************************* %
% ~~~~~~~~              Code snippets             ~~~~~~~~~ %
% ********************************************************* %

\usepackage{color} % para snipets de codigo coloreados
\usepackage{fancybox}  % para el sbox de los snipets de codigo

\definecolor{litegrey}{gray}{0.94}

\newenvironment{codesnippet}{%
	\begin{Sbox}\begin{minipage}{\textwidth}\sffamily\small}%
	{\end{minipage}\end{Sbox}%
		\begin{center}%
		\vspace{-0.4cm}\colorbox{litegrey}{\TheSbox}\end{center}\vspace{0.3cm}}



% ********************************************************* %
% ~~~~~~~~         Formato de las páginas         ~~~~~~~~~ %
% ********************************************************* %

\usepackage{fancyhdr}
\usepackage{parskip}
\pagestyle{fancy}

%\renewcommand{\chaptermark}[1]{\markboth{#1}{}}
\renewcommand{\sectionmark}[1]{\markright{\thesection\ - #1}}

\fancyhf{}

\fancyhead[LO]{Sección \rightmark} % \thesection\ 
\fancyfoot[LO]{\small{Luis Enrique Badell Porto, Nicolas Bukovits, Kevin Frachtenberg}}
\fancyfoot[RO]{\thepage}
\renewcommand{\headrulewidth}{0.5pt}
\renewcommand{\footrulewidth}{0.5pt}
\setlength{\hoffset}{-0.8in}
\setlength{\textwidth}{16cm}
%\setlength{\hoffset}{-1.1cm}
%\setlength{\textwidth}{16cm}
\setlength{\headsep}{0.5cm}
\setlength{\textheight}{25cm}
\setlength{\voffset}{-0.7in}
\setlength{\headwidth}{\textwidth}
\setlength{\headheight}{13.1pt}
\setlength{\parindent}{4em}
\setlength{\parskip}{\baselineskip}

\renewcommand{\baselinestretch}{1.1}  % line spacing

% ******************************************************** %


\begin{document}


\thispagestyle{empty}
\materia{Organización del Computador II}
\submateria{Primer Cuatrimestre de 2016}
\titulo{Trabajo Práctico 3}
\subtitulo{System Programming}
\integrante{Luis Enrique Badell Porto}{246/13}{luisbadell@gmail.com}
\integrante{Nicolas Bukovits}{546/14}{nicko_buk@hotmail.com}
\integrante{Kevin Frachtenberg}{247/14}{kevinfra94@gmail.com}
\grupo{Grupo: Yo no manejo el raiting, yo manejo un Rolls-Royce}
\maketitle

%{\small\textbf{\flushleft{Resumen}}\\
\abstract {En el siguiente trabajo pr\'actico, se realiz\'o una implementacion de un sistema operativo b\'asico en el cual se aplicaron los conceptos de System Programming vistos en clase. El sistema desarrollado corre en Bochs (programa que permite simular una computadora IBM-PC y realizar tareas de debugging) y permite correr tareas de usuario concurrentemente. La implementaci\'on del sistema fue realizada siguiendo una serie de ejercicios en los cuales se desarrolló de forma gradual el sistema completo. La resolucion de los ejercicios y las decisiones tomadas esta\'n explicadas en este informe.}

%\newpage

%\thispagestyle{empty}
%\vfill


\thispagestyle{empty}
\vspace{3cm}
\tableofcontents
\newpage


%\normalsize
\newpage



\section{Resolucion de Ejercicios}
\subsection{Ejercicio 1}

a) En este primera parte se empezó a completar la GDT, que contiene los descriptores globales del sistema. El archivo modificado fue el gtd.c que contiene el código de implementación de la GDT. Por normas de Intel, la primera entrada en la GDT tiene que ser nula, por lo que la misma se completó con todos sus valores en 0. Posteriormente se agregaron 4 segmentos requeridos por el enunciado: un segmento para código de nivel 0 (kernel), en el cual se configuró el atributo dpl (privilegio) en 0 para que sólo pueda ser usado por el Kernel, otro segmento para datos también de nivel 0 para el Kernel, un segmento de código de nivel 3 (usuario con dpl 3) y otro segmento de datos de nivel 3. Por requerimiento del enunciado todos estos selectores de segmento fueron configurados para que direccionen los primeros 878 MB de memoria, por lo que los atributos de límite y base son iguales para todos. Las diferencias que presentan son los atributos de dpl antes mencionados y del tipo de selector que fue cargado con 8 para código y 2 para datos.

b) En este punto del ejercicio se implementó el cambio de modo real a protegido y se seteó la pila del kernel en la dirección 0x27000. Este cambio fue realizado en el archivo kernel.asm que contiene el código asm que corresponde al kernel. El código del mismo comienza inhabilitando las interrupciones (mediante la instrucción cli) y cambiando el modo de video a 80 x 50. El código que se agregó para cumplir con el requerimiento en este punto fue habilitar A20, para lo cual sólo fue necesaria llamar mediante a un call a una función externa que ya estaba implementada y se encarga de dicha tarea y cargar el registro GTDR que contiene la direccción física en donde se encuentra la GDT. Para ello se utilizó la instrucción lgdt a la cual se le pasó como argumento la dirección de la gdt. Posteriormente fue necesario habilitar el modo protegido para lo cual era necesario que el bit de PE (que indica si el modo protegido está habilitado o no) del registro de control CR0 sea seteado en 1. Como no está permitido operar directamente con el registro CR0 se copió su valor al registro de propósito general eax mediante la instrucción MOV y posteriormente se realizó una operación OR del registro eax con el inmediatio de destino en 1, para que el último bit sea seteado en 1. Finalmente se copió el valor actualizado de eax en CR0 y se cambió el modo a protegido realizando un jump far. La instrucción utilizada fue jmp 0x20:mp donde 0x20 representa el selector de código de nivel 0. El cálculo realizado fue el siguiente: índice de código de nivel 0 (que es el 4) multiplicado por 8, es decir shifteado 3 lugares a la izuiqerda. El mp es una etiqueta definida en la próxima instrucción a ejecutar justo inmediatamente despúes del jump, en donde comienza el código que a partir de ese momento se va ejecutar en modo protegido. Las primeras instrucciones que se ejecutan en este punto se encargan de cargar los selectores de segmento que tienen que estar cargados porque es requerimiento del modo protegido. Se cargaron entonces los registros ds, es, gs, fs y ss con el índice de datos de nivel 0 de la GDT (segmentación flat). El cs (code segment no fue necesario cargarlo debido a que la instrucción jmp lo cargó. Se cargaron además los registros esp y ebp de la pila con el valor 0x27000 que es la dirección que indicaba el enunciado en donde está la pila del Kernel (MOV esp, 0x27000 y MOV ebp, 0x27000).

c) Se agregó un segmento adicional en la gdt, que describe el área de la pantalla en memoria que va a ser utilizada solamente por el kernel. El segmento que se agregó por lo tanto fue configurado con el dpl en 0 y es de tipo 2. La GDT entonces pasa a tener en este punto 6 segmentos (teniendo en cuenta al primer segmento nulo).

d) El enunciado solicita que se pinte el área del mapa en un fondo de color. Se implementó una macro en el archivo imprimir.mac denominada pintarPantalla la cual se encarga de pintar el fondo del mapa de color gris. El código de la misma es muy simple ya que se trata de un ciclo que copia a memoria en las posiciones del mapa, en los bytes de atributos de cada pixel, el valor correspondiente al color de fondo de gris. El código de la rutina es el siguiente: 

\begin{verbatim}
%macro pintarPantalla 0

    push ecx
    push edi

    mov ecx , 4000
    xor edi , edi
    %%cicloGris:
        mov word [edi*2 + 0x000B8000] , 0x7700  ; 
        inc edi
        loop %%cicloGris

    pop edi
    pop ecx

%endmacro
\end{verbatim}

\subsection{Ejercicio 2}

a) En esta parte de la implementación se tuvo que completar las entradas en la IDT (que contiene los descriptores de las interrupciones) para asociar diferentes rutinas a las exepciones del procesador. Los archivos modificados fueron el idt.c y el isr.asm. Se implementó la función idt_inicializar() en idt.c la cual se encarga de inicializar las excepciones. Las excepciones configuradas en este punto fueron las contenidas en el rango 0-19 y para configuraralas se definió una macro llamada IDT_ENTRY(numero) cuya implementación es la siguiente: 

\begin{verbatim}
#define IDT_ENTRY(numero)                                                                                        \
    idt[numero].offset_0_15 = (unsigned short) ((unsigned int)(&_isr ## numero) & (unsigned int) 0xFFFF);        \
    idt[numero].segsel = (unsigned short) GDT_OFF_IDX_DESC_CODE0;                                                                  \
    idt[numero].attr = (unsigned short) 0x8E00;                                                                  \
    idt[numero].offset_16_31 = (unsigned short) ((unsigned int)(&_isr ## numero) >> 16 & (unsigned int) 0xFFFF);
\end{verbatim}

La misma se encarga de definir en idt, que es un arreglo de 256 posiciones que contiene las descripciones de las exepciones, en la posicion indicada por numero, el offset, los atributos y el selector de segmento que se trata de un selector de código de nivel 0 ya que van a ser rutinas que va a correr el kernel. Dentro de idt_inicializar() sólo se tuvo que llamar 20 veces a IDT_ENTRY(i) siendo i el número de exepción comprendido en el rango de 0-19.
Por otro lado, el archivo irs.asm que contiene el código asm de las rutinas de atención de interrupciones fue modificado agregando una macro _isr: la cual es llamada cuando se produce una exepción y en la misma se usaba el número de exepción que se produjo para mostrar un mensaje en pantalla de la exepción. Para ello se definieron mensajes para cada una de las exepciones. Por ejemplo para el caso de la división por cero se agregó el siguiente mensaje:

\begin{verbatim}
msj0: db'Divide Error!'
msj0_len equ $ - msj0
\end{verbatim}

b) Se solicita que se implemente lo necesario para que se pueda utilizar la IDT y que se pruebe generando una exepción. Para ello en el archivo kernel.asm se realiza un CALL a la función idt_inicializar implementada en el punto anterior y luego se carga el registro IDTR que contiene la dirección física en donde se encuentra alojada la IDT. La instrucción para cargar la misma es lidt a la cual se le pasa como argumento la dirección de la idt. Con estas modificaciones el sistema ya estaba configurado para manejar las exepciones. Para probarlo se agregaron tres instrucciones cuyo propósito era dividir por cero para que se produzca la exepción de Divide Zero. El código agregado fue:

\begin{verbatim}
xor ebx, ebx
xor eax, eax
div eax
\end{verbatim}

El mismo produjo una exepción de dividir por cero, con lo que pudimos probar empíricamente que las exepciones estaban siendo manejadas y funcionaban correctamente. Dichas líneas de código fueron comentadas luego de que se corroboró que funcionaba para que no se produzca la exepción y el kernel siga ejecutando normalmente, y si eventualmente en algún momento se deseaba volver a verificar su funcionalidad sólo se tengan que descomentar. Además nos sirvió como documentación y registro de lo que se fue implementando. 



\subsection{Video y MMU}
%Paginacion y sarasa

\subsection{Interrupciones}
En esta parte, se completaron las interrupciones del reloj , la del teclado y la 0x66 que posteriomente sera usada en el juego
%(En el mundo de Star Wars, la orden 66 es la que usa el canciller Palpatine para iniciar la purga de los Jedi, creo deberiamos tener algun easter egg de SW en el tp) 
dado que la interrupcion del reloj la genera la CPU cada cierta cantidad tiempo, recien en este momento se habilitan las interrupciones mediante la instruccion STI.
El codigo de las interrupciones se encuentra en isr.asm 

\subsection{El retorno de la GDT y la aparicion de la TSS}

\subsection{El Scheduler}
Para realizar el scheduler, decidimos crear un struct tarea que contiene la posicion en el mapa,su indice en la GDT, su cr3, si tiene prendido el bit de presente en la GDT

\section{Conclusiones y trabajo futuro}


%\begin{codesnippet}
%\begin{verbatim}

%struct Pepe {

 %   ...

%};

%\end{verbatim}
%\end{codesnippet}

\end{document}

